\documentclass[11pt]{article}
\usepackage[margin=1in]{geometry}
\usepackage{graphicx}
\usepackage{caption}
\usepackage{hyperref}
\usepackage{booktabs}

\title{CiTonga Tone Production Analysis}
\author{David Nguyen \and Rosser Martin}
\date{\today}

\begin{document}
\maketitle

\begin{abstract}
This report examines how consonant context influences the realization of
High (H) and Low (L) tone F\textsubscript{0} values in Malawian CiTonga verb stems.
Two experiments compare pitch following sonorant vs. voiced‐obstruent onsets
(Experiment A) and H vs. L tones after individual voiced obstruents (Experiment B).
\end{abstract}

\section{Data \& Methods}
We analyze \texttt{citonga.csv}, focusing on:
\begin{itemize}
  \item \textbf{Tone}: H or L
  \item \textbf{C1.class}: sonorants vs.\ voiced obstruents
  \item \textbf{C1}: individual consonant (b, d, j, g, v, z)
  \item \textbf{meanf0}: mean pitch (Hz)
\end{itemize}

Plots are generated via \texttt{ggplot2} in \texttt{citonga\_tone\_analysis.R}.

\section{Experiment A: H Tone by C1 Class}

\subsection{Hypotheses}
\begin{description}
  \item[H$_0$:] No difference in mean F\textsubscript{0} of H tones after sonorants vs.\ voiced obstruents.
  \item[H$_1$:] A significant difference exists.
\end{description}

\subsection{Results}

\begin{figure}[h!]
  \centering
  \includegraphics[width=0.7\textwidth]{experimentA_boxplot.png}
  \caption{Mean F\textsubscript{0} of H Tone by C1 Class}
\end{figure}

\noindent
A two‐sample t‐test yields:\\
\texttt{t = 5.1426, df = 23, p = 3.28e-05}\\
\emph{p} < 0.05, so we reject H$_0$.

\section{Experiment B: Tone by Individual Voiced Obstruent}

\subsection{Visualization}

\begin{figure}[h!]
  \centering
  \includegraphics[width=0.7\textwidth]{experimentB_boxplot.png}
  \caption{Mean F\textsubscript{0} by Voiced Obstruent and Tone}
\end{figure}

\subsection{Statistical Test}

A t‐test comparing H vs.\ L after voiced obstruents yields:\\
\texttt{t = -5.4622, df = 153, p = 1.86e-07}\\
\emph{p} < 0.05, reject H$_0$.

\section{Discussion}
\begin{itemize}
  \item After sonorants, H tones are higher than after voiced obstruents.
  \item Unexpectedly, L tones after voiced obstruents surface at higher pitch than H tones.
  \item Future work: investigate aerodynamic or prosodic mechanisms driving L‐tone elevation.
\end{itemize}

\section*{References}
\begin{itemize}
  \item R Core Team (2023). R: A Language and Environment for Statistical Computing.
  \item Wickham H. (2016). \emph{ggplot2: Elegant Graphics for Data Analysis}.
\end{itemize}

\end{document}

